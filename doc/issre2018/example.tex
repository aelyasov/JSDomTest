%% --------------------------------------------------------------------
\section{Motivating Example}
\label{sec.example}
%% --------------------------------------------------------------------

\begin{figure}[!t]
  \begin{lstlisting}[style=htmlcssjs,language=JavaScript]
/*t dom */
function isGameFinished() {
 var obj = document.getElementById('sudoku'); |c \label{isGameFinished.getSudoku} |c
 var subDivs = obj.getElementsByTagName('DIV'); |c \label{isGameFinished.getDivs} |c
 var allOk = true;
 for (var no = 0; no < subDivs.length; no++) { |c \label{isGameFinished.inFor.begin} |c
  if (subDivs[no].className.indexOf('square') >= 0 |c \label{isGameFinished.if1.begin} |c
      && !subDivs[no].style.backgroundColor) {
   var spans=subDivs[no].getElementsByTagName('SPAN');
   if (spans[0].innerHTML != spans[1].innerHTML) { |c \label{isGameFinished.if2.begin} |c
    allOk = false; //target |c \label{isGameFinished.unfinished} |c
    break;
   }
  }
 } |c \label{isGameFinished.outFor} |c
  return allOk;
}
\end{lstlisting}
  \caption{JS function \texttt{isGameFinished.js} from sudoku}
  \label{code.isGameFinished}
\end{figure}

\begin{figure}[!t]
  \begin{lstlisting}[style=htmlcssjs, language=HTML5]
<!-- T1: (5,5) -->            <!-- T2: (5,6), (7,15) -->
<html>                        <html>
 <body>                        <body>
  <div id='sudoku'>             <div id='sudoku'>
                                 <div></div>
  </div>                        </div>
 </body>                       </body>
</html>                       </html>

<!-- T3: (7,8), (10,14) -->   <!-- T4: (7,8), (10,11) -->
<html>                        <html>
 <body>                        <body>
  <div id='sudoku'>             <div id='sudoku'>
   <div class='square'>          <div class='square'>
    <span></span>                 <span></span>
    <span></span>                 <span>TEST</span>
   </div>                        </div>
  </div>                        </div>
 </body>                       </body>
</html>                       </html>
  \end{lstlisting}
  \caption{Input DOM arguments for \texttt{isGameFinished.js}}
  \label{fig.isGameFinished.tests}
\end{figure}

Let us consider the JS function \texttt{isGameFinished} in Figure~\ref{code.isGameFinished} taken from the web game \emph{sudoku}~\cite{sudoku}. This function checks if a given sudoku solution is valid. First, it locates a DOM element \texttt{obj} representing the game field (line~\ref{isGameFinished.getSudoku}). Then, it collects all children \emph{DIV} elements of \texttt{obj} (line~\ref{isGameFinished.getDivs}). Each \emph{DIV} corresponds to an input square of the game (line~\ref{isGameFinished.if1.begin}) and consists of two \emph{SPAN} tags. The first \emph{SPAN} contains the value entered by the player, whereas the second one (hidden by default) stores the expected (correct) value for that square. In the for-loop in lines~\ref{isGameFinished.inFor.begin}-\ref{isGameFinished.outFor}, we iterate through the whole collection of the input squares. Once, we encounter a square with two unequal values (line~\ref{isGameFinished.if2.begin}), we terminate the loop and call the game unfinished (line~\ref{isGameFinished.unfinished}), otherwise finished.

Suppose now that we would like to unit test this function. We aim to maximize the branch coverage as a common testing criteria~\cite{zhu1997software}. That is, for each individual branch of the function under test (FUT) \texttt{isGameFinished}, we need to construct a test input that covers the branch and leads to the normal termination of the FUT. A JS function can take \emph{explicit} input arguments (in our case none), but it can also accept \emph{implicit} arguments such as a DOM state. In general, in order to produce a complete test for our function we have to construct both an appropriate DOM and input arguments. The tests should be decoupled from the application's HTML. Moreover, such HTML could still be under active development. Thus, the developers have to complement test with DOM fixtures.

%Our program has the following branches: $(6,15)$, $(6,7)$, $(7,9)$, $(7,14)$, $(10,11)$, $(10,13)$, $(16,17)$, $(16,18)$. Four tests in

Figure~\ref{fig.isGameFinished.tests} exhibits four tests that together provide the full branch coverage of the FUT. Let us, for example, take a look at how we can reach the target branch $(10,11)$. Line~\ref{isGameFinished.getSudoku} expects the DOM to have an element with an id \textquotesingle\texttt{sudoku}\textquotesingle. In order to enter the for-loop in line~\ref{isGameFinished.inFor.begin}, this element should have at least one child \emph{DIV} tag. The first if-condition in line~\ref{isGameFinished.if1.begin} requires each \emph{DIV} element to be of the class \textquotesingle\texttt{square}\textquotesingle\, and do not have a background color. The second if-condition (line~\ref{isGameFinished.if2.begin}) implies that the \emph{DIV} should consist of two \emph{SPAN} elements whose \emph{innerHtml} values are not equal. The resulting test \texttt{T4} in Figure~\ref{fig.isGameFinished.tests} meets all the above conditions.

As we have just witnessed, even for such a relatively simple example, the test data generation can be far from trivial procedure because it requires the deep understanding of the program semantics. In practice, web developers often have to deal with third-party code, including a large number of JS libraries freely available on the GitHub, which are commonly untested and poorly documented. At the same time, unit testing is the most fundamental level of testing and the easiest to implement. It saves the code from unexpected regressions and often documents the desired behavior of a program. Thus, in industry unit testing has became a standard practice in web development.
