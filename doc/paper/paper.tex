% This is "sig-alternate.tex" V2.1 April 2013
% This file should be compiled with V2.5 of "sig-alternate.cls" May 2012
%
% This example file demonstrates the use of the 'sig-alternate.cls'
% V2.5 LaTeX2e document class file. It is for those submitting
% articles to ACM Conference Proceedings WHO DO NOT WISH TO
% STRICTLY ADHERE TO THE SIGS (PUBS-BOARD-ENDORSED) STYLE.
% The 'sig-alternate.cls' file will produce a similar-looking,
% albeit, 'tighter' paper resulting in, invariably, fewer pages.
%
% ----------------------------------------------------------------------------------------------------------------
% This .tex file (and associated .cls V2.5) produces:
%       1) The Permission Statement
%       2) The Conference (location) Info information
%       3) The Copyright Line with ACM data
%       4) NO page numbers
%
% as against the acm_proc_article-sp.cls file which
% DOES NOT produce 1) thru' 3) above.
%
% Using 'sig-alternate.cls' you have control, however, from within
% the source .tex file, over both the CopyrightYear
% (defaulted to 200X) and the ACM Copyright Data
% (defaulted to X-XXXXX-XX-X/XX/XX).
% e.g.
% \CopyrightYear{2007} will cause 2007 to appear in the copyright line.
% \crdata{0-12345-67-8/90/12} will cause 0-12345-67-8/90/12 to appear in the copyright line.
%
% ---------------------------------------------------------------------------------------------------------------
% This .tex source is an example which *does* use
% the .bib file (from which the .bbl file % is produced).
% REMEMBER HOWEVER: After having produced the .bbl file,
% and prior to final submission, you *NEED* to 'insert'
% your .bbl file into your source .tex file so as to provide
% ONE 'self-contained' source file.
%
% ================= IF YOU HAVE QUESTIONS =======================
% Questions regarding the SIGS styles, SIGS policies and
% procedures, Conferences etc. should be sent to
% Adrienne Griscti (griscti@acm.org)
%
% Technical questions _only_ to
% Gerald Murray (murray@hq.acm.org)
% ===============================================================
%
% For tracking purposes - this is V2.0 - May 2012

\documentclass{sig-alternate}
\usepackage{tikz}
\usetikzlibrary{arrows,shadows,positioning}
\usepackage{pgf-umlsd}
\usepackage{tikz-uml}
\usepackage{hyperref}
\usepackage{upquote}
\usepackage{listings}
\usepackage{lsthaskell}
\usepackage{xspace}
\usepackage{algorithmic}
\usepackage[titlenotnumbered, linesnumbered, ruled]{algorithm2e} % language for algorithm description
\hypersetup{
  colorlinks   = true, %Colours links instead of ugly boxes
  urlcolor     = blue, %Colour for external hyperlinks
  linkcolor    = red, %Colour of internal links
  citecolor    = green %Colour of citations
}


%%%%%%%%%%%%%%%%%%%%%%%%%%%%%%%%%%%%%%%%%%%%%%%%%%%%%%%%%%%%%
\definecolor{ltblue}{rgb}{0,0.4,0.4}
\definecolor{dkblue}{rgb}{0,0.1,0.6}
\definecolor{dkgreen}{rgb}{0,0.4,0}
\definecolor{dkviolet}{rgb}{0.3,0,0.5}
\definecolor{dkred}{rgb}{0.5,0,0}

\definecolor{lightgray}{rgb}{0.95, 0.95, 0.95}
\definecolor{darkgray}{rgb}{0.4, 0.4, 0.4}
%\definecolor{purple}{rgb}{0.65, 0.12, 0.82}
\definecolor{editorGray}{rgb}{0.95, 0.95, 0.95}
\definecolor{editorOcher}{rgb}{1, 0.5, 0} % #FF7F00 -> rgb(239, 169, 0)
\definecolor{editorGreen}{rgb}{0, 0.5, 0} % #007C00 -> rgb(0, 124, 0)
\definecolor{orange}{rgb}{1,0.45,0.13}		
\definecolor{olive}{rgb}{0.17,0.59,0.20}
\definecolor{brown}{rgb}{0.69,0.31,0.31}
\definecolor{purple}{rgb}{0.38,0.18,0.81}
\definecolor{lightblue}{rgb}{0.1,0.57,0.7}
\definecolor{lightred}{rgb}{1,0.4,0.5}
% CSS
\lstdefinelanguage{CSS}{
  keywords={color,background-image:,margin,padding,font,weight,display,position,top,left,right,bottom,list,style,border,size,white,space,min,width, transition:, transform:, transition-property, transition-duration, transition-timing-function},	
  sensitive=true,
  morecomment=[l]{//},
  morecomment=[s]{/*}{*/},
  morestring=[b]',
  morestring=[b]",
  alsoletter={:},
  alsodigit={-}
}

% JavaScript
\lstdefinelanguage{JavaScript}{
  morekeywords={typeof, new, true, false, catch, function, return, null, catch, switch, var, if, in, while, do, else, case, break, for},
  morecomment=[s]{/*}{*/},
  morecomment=[l]//,
  morestring=[b]",
  morestring=[b]'
}

\lstdefinelanguage{HTML5}{
  language=html,
  sensitive=true,	
  alsoletter={<>=-},	
  morecomment=[s]{<!-}{-->},
  tag=[s],
  otherkeywords={
  % General
  >,
  % Standard tags
  <!DOCTYPE,
  </html, <html, <head, <title, </title, <style, </style, <link, </head, <meta, />,
  % body
  </body, <body,
  % Divs
  </div, <div, </div>, 
  % Paragraphs
  </p, <p, </p>,
  % scripts
  </script, <script,
  % More tags...
  <canvas, /canvas>, <svg, <rect, <animateTransform, </rect>, </svg>, <video, <source, <iframe, </iframe>, </video>, <image, </image>, <header, </header, <article, </article
  },
  ndkeywords={
  % General
  =,
  % HTML attributes
  charset=, src=, id=, width=, height=, style=, type=, rel=, href=,
  % SVG attributes
  fill=, attributeName=, begin=, dur=, from=, to=, poster=, controls=, x=, y=, repeatCount=, xlink:href=,
  % properties
  margin:, padding:, background-image:, border:, top:, left:, position:, width:, height:, margin-top:, margin-bottom:, font-size:, line-height:,
  % CSS3 properties
  transform:, -moz-transform:, -webkit-transform:,
  animation:, -webkit-animation:,
  transition:,  transition-duration:, transition-property:, transition-timing-function:,
  }
}

\lstdefinestyle{htmlcssjs} {%
  % General design
  backgroundcolor=\color{editorGray},
  basicstyle={\scriptsize\ttfamily},   
  frame=tb,
  % line-numbers
  xleftmargin={0.75cm},
  numbers=left,
  stepnumber=1,
  firstnumber=1,
  numberfirstline=true,	
  % Code design
  identifierstyle=\color{black},
  keywordstyle=\color{blue}\bfseries,
  ndkeywordstyle=\color{editorGreen}\bfseries,
  stringstyle=\color{editorOcher}\ttfamily,
  commentstyle=\color{brown}\ttfamily,
  % Code
  language=JavaScript,
  alsolanguage=HTML5,
  alsodigit={.:;},	
  tabsize=2,
  showtabs=false,
  showspaces=false,
  showstringspaces=false,
  extendedchars=true,
  breaklines=true,
  escapechar=|c
}
%%%%%%%%%%%%%%%%%%%%%%%%%%%%%%%%%%%%%%%%%%%%%%%%%%%%%%%%%%%%%

\listfiles % used to output the versions of all files being used by the latex

\newcommand{\Server}{\emph{Server}\xspace}
\newcommand{\Client}{\emph{Client}\xspace}
\newcommand{\User}{\emph{User}\xspace}
\newcommand{\Validator}{\emph{Validator}\xspace}

\begin{document}

% Copyright
\setcopyright{acmcopyright}
%\setcopyright{acmlicensed}
%\setcopyright{rightsretained}
%\setcopyright{usgov}
%\setcopyright{usgovmixed}
%\setcopyright{cagov}
%\setcopyright{cagovmixed}


% DOI
\doi{10.475/123_4}

% ISBN
\isbn{123-4567-24-567/08/06}

%Conference
\conferenceinfo{PLDI '13}{June 16--19, 2013, Seattle, WA, USA}

\acmPrice{\$15.00}

%
% --- Author Metadata here ---
\conferenceinfo{WOODSTOCK}{'97 El Paso, Texas USA}
%\CopyrightYear{2007} % Allows default copyright year (20XX) to be over-ridden - IF NEED BE.
%\crdata{0-12345-67-8/90/01}  % Allows default copyright data (0-89791-88-6/97/05) to be over-ridden - IF NEED BE.
% --- End of Author Metadata ---

\title{Search-Based Test Data Generation for JavaScript Functions on DOM}
%
% You need the command \numberofauthors to handle the 'placement
% and alignment' of the authors beneath the title.
%
% For aesthetic reasons, we recommend 'three authors at a time'
% i.e. three 'name/affiliation blocks' be placed beneath the title.
%
% NOTE: You are NOT restricted in how many 'rows' of
% "name/affiliations" may appear. We just ask that you restrict
% the number of 'columns' to three.
%
% Because of the available 'opening page real-estate'
% we ask you to refrain from putting more than six authors
% (two rows with three columns) beneath the article title.
% More than six makes the first-page appear very cluttered indeed.
%
% Use the \alignauthor commands to handle the names
% and affiliations for an 'aesthetic maximum' of six authors.
% Add names, affiliations, addresses for
% the seventh etc. author(s) as the argument for the
% \additionalauthors command.
% These 'additional authors' will be output/set for you
% without further effort on your part as the last section in
% the body of your article BEFORE References or any Appendices.

\numberofauthors{1}

\author{
  \alignauthor Alexander Elyasov, Wishnu Prasetya, Jurriaan Hage\\ 
  \affaddr{Utrecht University, The Netherlands}\\
  \email{\{a.elyasov,s.w.b.prasetya,j.hage\}@uu.nl}
}


\maketitle
\begin{abstract}
 
\end{abstract}


%
% The code below should be generated by the tool at
% http://dl.acm.org/ccs.cfm
% Please copy and paste the code instead of the example below. 
%
\begin{CCSXML}
TODO
\end{CCSXML}

\ccsdesc[500]{Computer systems organization~Embedded systems}
\ccsdesc[300]{Computer systems organization~Redundancy}
\ccsdesc{Computer systems organization~Robotics}
\ccsdesc[100]{Networks~Network reliability}

\printccsdesc

% We no longer use \terms command
%\terms{Theory}

\keywords{ACM proceedings; \LaTeX; text tagging}

\section{Introduction}
\label{sec.intro}
Recent empirical study of client-side JavaScript bugs~\cite{frolin:TSE16} has shown that 68\% of faults are DOM related.

Another empirical work~\cite{richards2010analysis}.

%% --------------------------------------------------------------------
\section{Motivating Example}
\label{sec.example}
%% --------------------------------------------------------------------

\begin{figure}[t]
  \begin{lstlisting}[style=htmlcssjs,language=JavaScript]
/*t dom */
function safeAdd() {
  var iframe = document.createElement("iframe");
  var whereto = document.getElementById("debug");
  var frameid = document.getElementById("frame");
  iframe.setAttribute("id", "frame");
  if (frameid) { 
    iframe.removeChild(frameid.childNodes[0]); |c \label{lbl:error} |c
    // frameid.parentNode.removeChild(frameid); |c \label{lbl:correction} |c
  } else { whereto.appendChild(iframe); } }  
  \end{lstlisting}
  \caption{Motivating Example}
  \label{fig..safeAdd.js}
\end{figure}


\begin{figure}[t]
  \begin{lstlisting}[style=htmlcssjs, language=HTML5]
<html>
  <body>
    <div id="frame"></div>
    <div id="debug"></div>
  </body>
</html>
  \end{lstlisting}

  \begin{lstlisting}[style=htmlcssjs, language=HTML5]
<html>
  <body>
    <div id="debug"></div>
  </body>
</html>
\end{lstlisting}  
\caption{Motivating Example}
\label{fig.safeAdd.html}
\end{figure}


\begin{figure}[t]
  \begin{lstlisting}[style=htmlcssjs,language=JavaScript]
/*t dom */
function revealAll() {
 for(var row=0; row<2; row++){
  for(var col=0; col<2; col++){
   var obj = document.getElementById('square_'+row+'_'+col);
   var spans = obj.getElementsByTagName('SPAN');
   spans[0].style.display='';
   spans[1].style.display='none';
   spans[1].style.color='#000000'; } } }
  \end{lstlisting}
  \caption{Motivating Example}
  \label{fig.revealAll.js}
\end{figure}

%% --------------------------------------------------------------------
\section{Test Generation Framework}
\label{sec.framework}
%% --------------------------------------------------------------------
The general workflow of our JavaScript test generation framework is represented as a sequence diagram in Figure~\ref{fig.framework.architect}. The diagram consists of four interacting components: \User, \Server\, \Validator and \Client, where \Server is the main one. It performs all preparatory steps (\emph{Initialization Phase}) as well as controls the data generation (\emph{Genetic Phase}). These two phases will be covered in details in Section~\ref{sub.sec.init.phase} and \ref{sub.sec.genetic.phase} respectively.

Every request sent to \Validator\footnote{\url{https://validator.w3.org/nu/}} with an HTML document as the body triggers a response containing the list of syntactic errors in that document. \Client is a \emph{Node.js}\footnote{\url{https://nodejs.org/en/}} application that is mainly responsible for the execution of the JavaScript code. It relies on the \emph{jsdom}\footnote{\url{https://github.com/tmpvar/jsdom}} library to simulate calls to native DOM. 

Let us get a closer look at the sequence of interactions in Figure~\ref{fig.framework.architect}. First, \User provides JS file (\emph{file.js}) that contains the function under test (FUT) together with its type and a configuration file \emph{config}. These data are processed by the \Server in the \emph{Initialization Phase}. As an output we get a tuple with the following information about FUT: control flow graph (\emph{cfg}), list of branches (\emph{branches}), static information about the program's constants (\emph{cinfo}), instrumented version of the original program (\emph{ifun}) and the function's signature (\emph{sig}). \Server sends a \texttt{POST} request to the \Client to store \emph{ifun} and \emph{sig}. Second, \Server asks \User to select a branch(s) that has to be covered by the execution of FUT. For each branch our framework triggers the \emph{Genetic Phase (Loop)}. Inside of this phase it starts with a randomly generated population (\emph{Random Loop}), and continues to improve it until earthier perfect value is found or generation limit is reached. Each candidate in the population has to be evaluated. \Client parses the candidate according the know signature, invokes \emph{ifun} on the result, collects some information along the execution and sends it back to \Server, where the final fitness value is computed for that candidate (\emph{Evaluation Loop}). New population is the formed as joint result of the \emph{Crossover} and \emph{Mutation Loop}. At the end of the \emph{Fitness Loop} the best candidate with the best fitness value is reported to \User. After exhausting all branches \emph{Genetic Loop} terminates. 

\begin{figure}
  \centering 
  \begin{sequencediagram}[font=\scriptsize]
    \newinst{user}{User}
    \newinst{server}{Server}
    \newinst[1]{validator}{Validator}
    \newinst{client}{Client}
    
    \begin{call}{user}{file.js + config}{server}{\textbf{end}}
      \begin{call}{server}{Initialize(file.js, config)}{server}{(cfg, branches, cinfo, ifun, sig)}
      \end{call}
      
      \begin{call}{server}{POST /init \{ifun, sig\}}{client}{status}
      \end{call}
      
      \begin{call}{server}{Select(branches)}{user}{$\text{branches}'$}       
      \end{call}
      
      \begin{sdblock}{Genetic Loop}{\hspace{2mm}$\text{branch} \in \text{branches}'$}
        \begin{sdblock}{Random Loop}{errors > 0}
          \begin{call}{server}{pop = Random(cinfo, config)}{validator}{errors}
          \end{call}
          \prelevel
        \end{sdblock}
        
        \begin{sdblock}{Fitness Loop}{(fitness $\neq$ 0) $\vee$ gen.limit}

          \begin{sdblock}{Evaluation Loop}{p $\in$ pop}
            \begin{call}{server}{POST /genetic \{p\}}{client}{response = ifun(p: sig)}
            \end{call}
            
            \begin{callself}{server}{Evaluate(branch, cfg, response)}{fitness}
            \end{callself}
            \prelevel
          \end{sdblock}
          
          \begin{sdblock}{Crossover Loop}{errors > 0}
            \begin{call}{server}{cross = Crossover(pop)}{validator}{errors}
            \end{call}
            \prelevel
          \end{sdblock}

          \begin{sdblock}{Mutation Loop}{errors > 0}
            \begin{call}{server}{mut = Mutation(cinfo, pop)}{validator}{errors}
            \end{call}
            \prelevel
          \end{sdblock}
         
          \begin{callself}{server}{cross $\cup$ mut}{pop}
          \end{callself}
          \prelevel
        \end{sdblock}
        \mess{server}{best fitness}{user}
        
      \end{sdblock}
    \end{call}
  \end{sequencediagram}
  \caption{Architecture of the test framework}
  \label{fig.framework.architect} 
\end{figure}

\begin{algorithm}[t!]
  \caption{Initialization Phase}
  \label{alg.init}
  \algsetup{linenosize=\tiny}
  \scriptsize
  \DontPrintSemicolon
  \SetAlgoVlined
  \SetKwInOut{Input}{Input}
  \SetKwInOut{Output}{Output}
  \SetKwInOut{Require}{Require}
  \SetKwProg{Fn}{Function}{}{}
  \Input{
    $fun.js$ is a JS file with FUT and type annotation
  }
  \Output{
    tuple with $(cfg, branches, cinfo)$
  }
  \Fn{$Initialize(fun.js)$}{
 
    $(ast, sig) \longleftarrow ParseFunAndSig(fun.js)$\label{alg1.parse.fun.sig}\;
    $ast' \longleftarrow TransAST(ast)$\;
    $cfg \longleftarrow BuildCFG(ast')$\;
    $branches \longleftarrow GetBranches(ast')$\;
    $cinfo \longleftarrow GetConstInfo(ast')$\;
    $ifun \longleftarrow Instrument(ast')$\;
    $Send(POST, /init, \{ifun, sig\})$\;
    \Return{$(cfg, branches, cinfo)$}
    }
\end{algorithm}

\begin{algorithm}[!t]
  \caption{Genetic Phase}
  \label{alg.gen}
  \algsetup{linenosize=\tiny}
  \scriptsize
  \DontPrintSemicolon
  \SetAlgoVlined
  \SetKwInOut{Input}{Input}
  \SetKwInOut{Output}{Output}
  \SetKwInOut{Require}{Require}
  \SetKwProg{Fn}{Function}{}{}
  \Fn{$Genetic(b_N, cfg, cinfo, config)$}{
    $(size_P, size_A, size_G, rate_C, rate_M) \longleftarrow Read(config)$\;
    $archive \longleftarrow \emptyset$\;
    $i \longleftarrow 0 $\;
    $pop \longleftarrow Random(cinfo, size_P)$\;
    \Repeat{$isPerfect(f^*) \vee i > size_G $}{
      $scoredPop \longleftarrow \emptyset$\;
      \ForAll{$p \in pop$}{
        $(tr, d_B, d_L, consts) \longleftarrow Send(POST, /genetic, \{p\})$\;
        \tcp{evaluate $ifun(p: sig)$}
        $(f, p) \longleftarrow Evaluate(b_N, cfg, tr, d_B, d_L)$\;
        $scoredPop \longleftarrow (f, p) \cup scoredPop$\;
        $cinfo \longleftarrow consts \cup cinfo$\;
      }
      $combo \longleftarrow scoredPop \cup archive$\;
      $cross \longleftarrow Crossover(combo, rate_C)$\;
      $mut \longleftarrow Mutation(combo, cinfo, rate_M)$\;
      $pop \longleftarrow cross \cup mut$\;
      $archive \longleftarrow GetBest(combo, size_A)$\;
      $(f^*, p^*) \longleftarrow head(archive)$\;
      $i \longleftarrow i + 1$\;
    }
    \Return{$(f^*, p^*)$} 
  }
\end{algorithm} 

\subsection{Initialization Phase}
\label{sub.sec.init.phase}

\subsection{Genetic Phase}
\label{sub.sec.genetic.phase}

The heart of our test data generation framework is its genetic phase. The main workflow of the underlying genetic algorithm (GA) is quite standard~\cite{poli2008field}. First, random population is generated. Then, this population is evaluated against given fitness function. If perfectly fitted candidate is found or the budget is exhausted, the algorithm terminates and reports the best candidate so far. Otherwise it enters an iterative phase where after a selections procedure some candidates are crossed over and others are mutated. Afterwards a new population is formed and the evaluation repeats.

In order to instantiate genetic algorithm for a particular problem, its user has to define all the aforementioned operations specific for the problem in question. Since GAs have been commonly used for generation of test data for primitive types such as integers and strings~\cite{paper_on_search-based_test_case_generation}, we skip their definitions and rather focus or the interesting case, namely, HTML generation.  

\subsubsection{Random HTML Generation}
\label{sub.sub.sec.random.html}

We want to generate HTML as it is specified by WHATWG\footnote{\url{https://html.spec.whatwg.org}}. HTML is an XML-family markup language that uses special tags and attributes to describe document structures on the Web. Recent HTML5 standard defines more than hundred of tags and attributes which both have to comply with extremely complex logic to constitute a syntactically valid HTML. Browsers are usually tolerant to syntactic errors in HTML. They will silently fix them and display the page anyway.

In order to support the variability of HTML content, we have developed an eDSL that prescribes the generation of syntactically valid HTML. This was a challenge on its own because, as we have mentioned early, the complexity of HTML and lack of formal specification. Our language is inspired by QuickCheck~\cite{claessen2011quickcheck} and essentially employs the library\footnote{\url{https://hackage.haskell.org/package/QuickCheck}} to define an \texttt{Arbitrary} instance for the \texttt{Html} data type. With some technical details left out for the sake of simplicity, Figure~\ref{fig.html.arb.def} highlights that definition. Data type \texttt{HtmlState} (lines~\ref{ref.HtmlState.begin}--\ref{ref.HtmlState.end}) describes the internal state of \texttt{Html} during generation. The state is composed of the following components:
\begin{itemize}
\item \texttt{depth} and \texttt{degree} specify maximum depth and node degree of HTML respectively
\item \texttt{ctx} is the current value of the HTML content model stack 
\item \texttt{tagFreq} provides statistical information about the frequency of tags in HTML
\item \texttt{tags}, \texttt{ids} and \texttt{classes} are predefined sets of HTML elements which are given preference in the generation process.  
\end{itemize}
Combined with the random generation monad \texttt{Gen}, this state is wrapped into a state monad transformer~\cite{jones1995functional} (line~\ref{ref.GenState}) which returns the randomly generated \texttt{Html} (line~\ref{ref.GenHtmlState}). Each HTML tag should be defined as a value of type \texttt{GenHtmlState}. Lines~\ref{ref.generateHTML.begin}--\ref{ref.generateHTML.end} show an example of content generation for the \texttt{<html>} tag. According to its specification\footnote{\url{https://html.spec.whatwg.org/multipage/semantics.html\#the-html-element}}, the tag is composed of the \texttt{<head>} tag followed by the \texttt{<body>} tag. First, we store the state of HTML generation and retrieve the current the depth (lines~\ref{ref.generateHTML.getState} and \ref{ref.generateHTML.getDepth} respectively). Then, if depth is equal to zero, we just return some string constant (line~\ref{ref.generateHTML.returnString}). Otherwise, we update the state with the depth decreased by one, generate new content for \texttt{head} and \texttt{body}, and return their combination (lines~\ref{ref.generateHTML.updateState}, \ref{ref.generateHTML.newHead}, \ref{ref.generateHTML.newBody} and \ref{ref.generateHTML.end} respectively).

\begin{figure}[t]
  \begin{haskell}
instance Arbitrary Html where
  arbitrary = evalStateT generateHTML defaultState

generateHTML :: GenHtmlState |c \label{ref.generateHTML.begin} |c
generateHTML = do
  state <- get |c \label{ref.generateHTML.getState} |c   
  let d = getDepth state |c \label{ref.generateHTML.getDepth} |c
  if d == 0
  then lift \dollar return \dollar toHtml "HTML" |c \label{ref.generateHTML.returnString} |c
  else do put state{depth = d - 1} |c \label{ref.generateHTML.updateState} |c
           head <- generateHEAD |c \label{ref.generateHTML.newHead} |c
           body <- generateBODY |c \label{ref.generateHTML.newBody} |c
           lift\dollar return \dollar docTypeHtml \dollar head >> body |c \label{ref.generateHTML.end} |c

data Tag = HTML | HEAD | BODY | ...
type Context = CFlow | CMetadata | ...
data HtmlState = |c \label{ref.HtmlState.begin} |c
  HtmlState { depth      :: Int 
             , degree    :: Int    
             , ctx       :: Stack Context 
             , tagFreq   :: [(Int, Tag)] 
             , tags      :: [Tag]
             , ids       :: [String]
             , classes   :: [String] } |c \label{ref.HtmlState.end} |c
type GenState s = StateT s Gen |c \label{ref.GenState} |c
type GenHtmlState = GenState HtmlState Html |c \label{ref.GenHtmlState} |c
  \end{haskell}
  \caption{\texttt{Arbitrary} instance for \texttt{Html} data type}
  \label{fig.html.arb.def}
\end{figure}

\subsubsection{Genetic HTML Operations}
\label{sub.sub.sec.genetic.oper}

In essence, HTML document is a labeled tree with \emph{tags} instead of nodes and \emph{attributes} as labels. Therefore crossover and mutations are defined similar to those concepts for the expression trees in genetic programming.  

\textbf{HTML Crossover} takes two HTML trees, randomly picks a node in each tree, and returns the first tree with a sub-tree swapped from the second tree.

\textbf{HTML Mutations} are of three categories: \emph{NewTree}, \emph{DropTree}, and \emph{ShuffleAttributes}. \emph{NewTree} simply generate a new tree based on the given state of constant pool. \emph{DropTree} mutation selects a node in a tree and drops the sub-tree starting from that node. \emph{ShuffleAttributes} for a given tree and an attribute type, re-assigns the attributes of the specified type on that tree. Currently we instantiated the \emph{ShuffleAttributes} mutation for only for two types of attributes \emph{class} and \emph{id}, which gave us it total four variations for mutation. Each time a mutation is performed the type is chosen uniformly. In the future we consider experiment with assigning some priorities to curtain mutations.   

\subsubsection{Fitness Function}
\label{sub.sub.sec.fitness.fun}


%% --------------------------------------------------------------------
\section{Empirical Evaluation}
\label{sec.evaluation}
%% --------------------------------------------------------------------

\begin{description}
\item[\textbf{Q1}] How often \emph{validator} rejects generated HTML in case of the random and genetic (mutation and crossover) phases. This question characterizes potential performance issues that can occur in the generation.     
\end{description}

\subsection{Case Studies}
\label{sub.sec.case.studies}

Consider examples from: \cite{artemis2011}, \cite{dom2011}...

%% --------------------------------------------------------------------
\section{Related Work}
\label{sec:related.work}
%% --------------------------------------------------------------------

\subsection{Search-Based Test Case Generation}
\label{sub.sec.search.based}

\subsection{Static Analysis of JavaScript}
\label{sub.sec.js.static.anal}

Static analysis is a powerful technique commonly used as solid alternative for program testing. In fact both techniques can benefit from one another. For such weakly typed and dynamic languages as JavaScript employing static analysis to infer types can be an invaluable source of information to identify potential bugs or even guarantee their absence. Jensen et al.~\cite{tajs2009} presented such an analysis for JavaScript using abstract interpretation. In the follow-up work~\cite{dom2011} they extended the analysis with the abstractions to reason about HTML DOM and browser API. Given step let the authors to carry static analysis at the level of complete web application (JavaScript + HTML).
\cite{jquery2014}


\subsection{Whole Web Application Testing}
\label{sub.sec.web.app.test}

Artemis~\cite{artemis2011} is a feedback-directed random testing tool for JavaScript web application. It discoverers new tests by generating sequences of executable events and monitoring they effect on the state of the application. \cite{ail2013}

\subsection{JavaScript Unit Testing}
\label{sub.sec.js.unit.test}

Generating fixture for JavaScript unit tests by means of concolic execution~\cite{amin:ase15}. Inference of UI oracles to complement JS unit tests~\cite{icst16}.

\subsection{Misc JavaScript Testing Techniques}
\label{sub.sec.misc.test.tech}

Discover patterns from bug fixes and imposing them as potential bugs in the future~\cite{quinn:fse16}. 

\section{Conclusion}
\label{sec:concl}
1 page.

%
% The following two commands are all you need in the
% initial runs of your .tex file to
% produce the bibliography for the citations in your paper.
\bibliographystyle{abbrv}
\bibliography{sigproc}  % sigproc.bib is the name of the Bibliography in this case
% You must have a proper ".bib" file
%  and remember to run:
% latex bibtex latex latex
% to resolve all references
%
% ACM needs 'a single self-contained file'!
%

%\balancecolumns

\end{document}
